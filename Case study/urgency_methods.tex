% Options for packages loaded elsewhere
\PassOptionsToPackage{unicode}{hyperref}
\PassOptionsToPackage{hyphens}{url}
%
\documentclass[
]{article}
\usepackage{amsmath,amssymb}
\usepackage{iftex}
\ifPDFTeX
  \usepackage[T1]{fontenc}
  \usepackage[utf8]{inputenc}
  \usepackage{textcomp} % provide euro and other symbols
\else % if luatex or xetex
  \usepackage{unicode-math} % this also loads fontspec
  \defaultfontfeatures{Scale=MatchLowercase}
  \defaultfontfeatures[\rmfamily]{Ligatures=TeX,Scale=1}
\fi
\usepackage{lmodern}
\ifPDFTeX\else
  % xetex/luatex font selection
\fi
% Use upquote if available, for straight quotes in verbatim environments
\IfFileExists{upquote.sty}{\usepackage{upquote}}{}
\IfFileExists{microtype.sty}{% use microtype if available
  \usepackage[]{microtype}
  \UseMicrotypeSet[protrusion]{basicmath} % disable protrusion for tt fonts
}{}
\makeatletter
\@ifundefined{KOMAClassName}{% if non-KOMA class
  \IfFileExists{parskip.sty}{%
    \usepackage{parskip}
  }{% else
    \setlength{\parindent}{0pt}
    \setlength{\parskip}{6pt plus 2pt minus 1pt}}
}{% if KOMA class
  \KOMAoptions{parskip=half}}
\makeatother
\usepackage{xcolor}
\usepackage[margin=1in]{geometry}
\usepackage{color}
\usepackage{fancyvrb}
\newcommand{\VerbBar}{|}
\newcommand{\VERB}{\Verb[commandchars=\\\{\}]}
\DefineVerbatimEnvironment{Highlighting}{Verbatim}{commandchars=\\\{\}}
% Add ',fontsize=\small' for more characters per line
\usepackage{framed}
\definecolor{shadecolor}{RGB}{248,248,248}
\newenvironment{Shaded}{\begin{snugshade}}{\end{snugshade}}
\newcommand{\AlertTok}[1]{\textcolor[rgb]{0.94,0.16,0.16}{#1}}
\newcommand{\AnnotationTok}[1]{\textcolor[rgb]{0.56,0.35,0.01}{\textbf{\textit{#1}}}}
\newcommand{\AttributeTok}[1]{\textcolor[rgb]{0.13,0.29,0.53}{#1}}
\newcommand{\BaseNTok}[1]{\textcolor[rgb]{0.00,0.00,0.81}{#1}}
\newcommand{\BuiltInTok}[1]{#1}
\newcommand{\CharTok}[1]{\textcolor[rgb]{0.31,0.60,0.02}{#1}}
\newcommand{\CommentTok}[1]{\textcolor[rgb]{0.56,0.35,0.01}{\textit{#1}}}
\newcommand{\CommentVarTok}[1]{\textcolor[rgb]{0.56,0.35,0.01}{\textbf{\textit{#1}}}}
\newcommand{\ConstantTok}[1]{\textcolor[rgb]{0.56,0.35,0.01}{#1}}
\newcommand{\ControlFlowTok}[1]{\textcolor[rgb]{0.13,0.29,0.53}{\textbf{#1}}}
\newcommand{\DataTypeTok}[1]{\textcolor[rgb]{0.13,0.29,0.53}{#1}}
\newcommand{\DecValTok}[1]{\textcolor[rgb]{0.00,0.00,0.81}{#1}}
\newcommand{\DocumentationTok}[1]{\textcolor[rgb]{0.56,0.35,0.01}{\textbf{\textit{#1}}}}
\newcommand{\ErrorTok}[1]{\textcolor[rgb]{0.64,0.00,0.00}{\textbf{#1}}}
\newcommand{\ExtensionTok}[1]{#1}
\newcommand{\FloatTok}[1]{\textcolor[rgb]{0.00,0.00,0.81}{#1}}
\newcommand{\FunctionTok}[1]{\textcolor[rgb]{0.13,0.29,0.53}{\textbf{#1}}}
\newcommand{\ImportTok}[1]{#1}
\newcommand{\InformationTok}[1]{\textcolor[rgb]{0.56,0.35,0.01}{\textbf{\textit{#1}}}}
\newcommand{\KeywordTok}[1]{\textcolor[rgb]{0.13,0.29,0.53}{\textbf{#1}}}
\newcommand{\NormalTok}[1]{#1}
\newcommand{\OperatorTok}[1]{\textcolor[rgb]{0.81,0.36,0.00}{\textbf{#1}}}
\newcommand{\OtherTok}[1]{\textcolor[rgb]{0.56,0.35,0.01}{#1}}
\newcommand{\PreprocessorTok}[1]{\textcolor[rgb]{0.56,0.35,0.01}{\textit{#1}}}
\newcommand{\RegionMarkerTok}[1]{#1}
\newcommand{\SpecialCharTok}[1]{\textcolor[rgb]{0.81,0.36,0.00}{\textbf{#1}}}
\newcommand{\SpecialStringTok}[1]{\textcolor[rgb]{0.31,0.60,0.02}{#1}}
\newcommand{\StringTok}[1]{\textcolor[rgb]{0.31,0.60,0.02}{#1}}
\newcommand{\VariableTok}[1]{\textcolor[rgb]{0.00,0.00,0.00}{#1}}
\newcommand{\VerbatimStringTok}[1]{\textcolor[rgb]{0.31,0.60,0.02}{#1}}
\newcommand{\WarningTok}[1]{\textcolor[rgb]{0.56,0.35,0.01}{\textbf{\textit{#1}}}}
\usepackage{graphicx}
\makeatletter
\def\maxwidth{\ifdim\Gin@nat@width>\linewidth\linewidth\else\Gin@nat@width\fi}
\def\maxheight{\ifdim\Gin@nat@height>\textheight\textheight\else\Gin@nat@height\fi}
\makeatother
% Scale images if necessary, so that they will not overflow the page
% margins by default, and it is still possible to overwrite the defaults
% using explicit options in \includegraphics[width, height, ...]{}
\setkeys{Gin}{width=\maxwidth,height=\maxheight,keepaspectratio}
% Set default figure placement to htbp
\makeatletter
\def\fps@figure{htbp}
\makeatother
\setlength{\emergencystretch}{3em} % prevent overfull lines
\providecommand{\tightlist}{%
  \setlength{\itemsep}{0pt}\setlength{\parskip}{0pt}}
\setcounter{secnumdepth}{-\maxdimen} % remove section numbering
\ifLuaTeX
  \usepackage{selnolig}  % disable illegal ligatures
\fi
\IfFileExists{bookmark.sty}{\usepackage{bookmark}}{\usepackage{hyperref}}
\IfFileExists{xurl.sty}{\usepackage{xurl}}{} % add URL line breaks if available
\urlstyle{same}
\hypersetup{
  pdftitle={Urgency Methods},
  pdfauthor={Henrique Sposito},
  hidelinks,
  pdfcreator={LaTeX via pandoc}}

\title{Urgency Methods}
\author{Henrique Sposito}
\date{2024-04-28}

\begin{document}
\maketitle

\hypertarget{urgency-theorymethods}{%
\section{Urgency theory/methods}\label{urgency-theorymethods}}

\hypertarget{political-promises}{%
\subsection{Political promises}\label{political-promises}}

Politicians make promises to gather and maintain political support.
Political promises are statements delivered by politicians expressing
their intent to take or continue a political action in the future. These
promises inform electorates about what politician intend to do and can
influence electoral behavior (see
\href{https://www.jstor.org/stable/pdf/1827369.pdf}{Downs 1957}). When,
and the extent to which, politicians keep or not their political
promises has been widely studied in political science (see
\href{https://ejpr.onlinelibrary.wiley.com/doi/abs/10.1111/1475-6765.12173}{Horn
and Jensen 2017};
\href{https://onlinelibrary.wiley.com/doi/10.1111/ajps.12789}{Schneider
and Thomson 2023}). However, besides issues related to how promise
keeping/breaking is measured (see cite), identifying promises is
challenging in the first place. Politicians make promises in discourse
but data on political promises is often collected from party manifestos
(e.g.~the
\href{https://manifesto-project.wzb.eu/information/documents/cmp_emp_mapping}{Manifesto
Research on Political Representation} data). This means promises coded
in research are aggregated at the party level and is often disconnected
from the actual promises made by individual politicians. As well,
promises are not always explicit in discourse. Politicians can be vague
about their future intentions
(\href{https://ejpr.onlinelibrary.wiley.com/doi/abs/10.1111/1475-6765.12173}{Horn
and Jensen 2017}). Ambiguity regarding political positions and promises
can be strategic for politicians
(\href{https://www.cambridge.org/core/services/aop-cambridge-core/content/view/D0D301CD53950B2D7C9BC6B492D413B1/S2049847016000182a.pdf/strategic-ambiguity-of-party-positions-in-multi-party-competition.pdf}{Brauninger
and Giger 2016}). Moreover, research frequently aggregate promises into
broad categories (e.g.~social welfare) but disregard how they relate to
one another (e.g.~which promises are more urgent), across categories
(e.g.~how promises related the economy differ from promises about
health), or in relation to politicians (e.g.~do certain politicians make
more promises).

Promises made by politicians are categorized and ranked by audiences.
Political discourses are fundamentally different than other types of
discourses as they are usually unstructured and cover multiple topics
(\href{https://www.cambridge.org/core/services/aop-cambridge-core/content/view/F7AAC8B2909441603FEB25C156448F20/S1047198700013401a.pdf/div-class-title-text-as-data-the-promise-and-pitfalls-of-automatic-content-analysis-methods-for-political-texts-div.pdf}{Grimmer
and Stewart 2013}). As well, politicians are not necessarily coherent or
consistent over time, across settings, or when it comes to their
ideological commitments when speaking (see
\href{https://www.tandfonline.com/doi/full/10.1080/09644016.2023.2220639}{Silva-Muller
and Sposito 2023}). However, there are certain common aspects of `doing
politics' that politicians must perform in discourse (see
\href{https://e-l.unifi.it/pluginfile.php/909651/mod_resource/content/1/Van\%20Dijk\%20Waht\%20is\%20political\%20discourse\%20analysis.pdf}{Dijk
1997}), such as making future promises. Whereas politicians can be
ambiguous when it comes to future actions, promises in discourse vary in
how pressing they appear. That is, vagueness about future promises
provides audiences with important information about how urgent specific
future actions might be (i.e.~vague promises appear less urgent than
precise promises). Urgency relates to the degree in which future
political action is necessary and immediate. Audiences infer the
ordering of future actions proposed by politicians based on how these
are constructed in discourse. Grasping the degree of urgency related to
promises allows to compare politicians' preferences related to future
action. By extracting future preferences on expressed promises, rather
than underlying preferences based on party manifestos, we can
systematically investigate how, when, and where politicians make
promises and how these relate to preferences stated elsewhere
(e.g.~party manifestos). This article develops a new method for studying
future promises in political discourses, namely urgency analysis.
Urgency analysis does not assume a specific interpretation of the
meanings associated with discourse, rather it focuses on how urgent
promises about future actions expressed in discourses are.

\hypertarget{urgency-analysis}{%
\subsection{2. Urgency Analysis}\label{urgency-analysis}}

Methodological choices made by social scientists studying political
discourses can be inadequate and consequential. Besides borrowing text
analysis methods from other disciplines without considering how
political discourses are unique (see
\href{https://www.cambridge.org/core/services/aop-cambridge-core/content/view/F7AAC8B2909441603FEB25C156448F20/S1047198700013401a.pdf/div-class-title-text-as-data-the-promise-and-pitfalls-of-automatic-content-analysis-methods-for-political-texts-div.pdf}{Grimmer
and Stewart 2013}), studies in political science often suppose a
specific interpretation of the meanings attributed to text. Instead,
urgency analysis employs a combination of Natural Language Process
(NLP), topic modelling, and dictionary approaches to text analysis to
extract and ranks promises in political discourses. Urgency analysis
provide a localized (section and issue specific), normalized (comparable
by default), and inclusive (accounts for metadata as setting and timing)
new method for the analysis of political discourses. Urgency analysis is
implemented for R with the ``poldis'' package, making it an easy, free,
and accessible tool for researchers interested in analyzing political
discourses. Below we detail how urgency analysis works.

Urgency analysis identifies and codes promises related to future action
in political discourses. The first step in urgency analysis is to
annotate texts. In practice, urgency analysis ``stands on the shoulders
of giants'' by relying on
\href{https://spacy.io/universe/project/spacyr}{spacy(r)} NLP algorithm
to annotate texts. Text annotations are offered at the word (token) and
sentence levels. That is, syntax metadata is extracted for words or
sentences in political texts. The syntax metadata helps to code words or
sentences when it makes sense rather than match all instances (e.g.~code
adverbs or adjectives where/when present).

\begin{Shaded}
\begin{Highlighting}[]
\FunctionTok{library}\NormalTok{(poldis)}
\NormalTok{words }\OtherTok{\textless{}{-}} \FunctionTok{annotate\_text}\NormalTok{(sample\_text}\SpecialCharTok{$}\NormalTok{text)}
\NormalTok{sentences }\OtherTok{\textless{}{-}} \FunctionTok{annotate\_text}\NormalTok{(sample\_text}\SpecialCharTok{$}\NormalTok{text, }\AttributeTok{level =} \StringTok{"sentences"}\NormalTok{)}
\end{Highlighting}
\end{Shaded}

Once annotated, promises can be extracted. Promises are captured by
looking at sentences containing modal verbs and adverbs (e.g.~must) or
other indications of future looking sentences based on dictionary
(e.g.~going to). At the same time, sentences which include present
actions (e.g.~are doing) or negative sentences (e.g.~should not do) are
excluded. This is done to avoid capturing actions that are taking place
and are, therefore, not promises (i.e.~politicians taking stock) or
sentences that appear urgent but refer to actions that will not be taken
(i.e.~talk about things they will never do/negative promises). Promises
are coded at the sentence level because, as politicians often speak
about multiple topics in discourse, extracting sentences allow us to be
sure that we are capturing urgency for a specific promise. When promises
refer to the same object in nearby sentences, they are merged. This is
done by identifying sentences that are too short or start with sentence
connectors (e.g.~first, second). Coding at the sentence level errs on
the side of caution. Although we might miss some cases where urgency is
declared in subsequent sentences or where deeper interpretation is
required, we prefer to be sure that we are capturing urgency for
specific promises (i.e.~avoid false negatives).

\begin{Shaded}
\begin{Highlighting}[]
\NormalTok{promises }\OtherTok{\textless{}{-}} \FunctionTok{extract\_promises}\NormalTok{(sentences)}
\end{Highlighting}
\end{Shaded}

With promises extracted, we can code their urgency. To grasp with the
degree in which future political action is necessary and immediate, we
code four dimensions of urgency using purpose-built dictionaries of
terms developed explicitly for each of these dimensions. On the one
hand, in relation to the immediacy of a promise, we code the frequency
and the timing of a promise. The frequency refers to reiteration of the
topic, solution, or context in a promise. This implies reiterated
promises are more urgent (e.g.~we must act constantly to vs.~sometimes
we must act). Relatedly, timing refers to the time orientation of a
promise. This implies that short term promises are more urgent than long
term (e.g.~we must act now vs.~we must act in the future). On the other
hand, in relation to the necessity of a promise, we code the degree and
the commitment levels of a promise. Degree refers to the level at which
a promise is urgent. This implies more urgent promises come accompanied
with stronger degree adjectives and adverbs (e.g.~extremely important
vs.~rather important). Degree adjectives and adverbs are scored with the
help of the SO-CALL dictionary (cite). Similarly, the commitment level
of a promise refers to the intensity of the promise commitment. This
implies a stronger level of commitment to action means a promise is more
urgent (e.g.~we should act vs.~we must act). Once each of these
dimensions are scored based on the purpose-built dictionary of terms,
scores are added and normalized by the number of words in the promise
they appear in. This approach to normalization is purposely biased
towards short promises. We believe that when politicians make concise
and strong promises, these appear more urgent to audiences than promises
embedded in longer sentences. Though, urgency scores are also available
for each of the dimensions separately or not normalized if users prefer
such.

\begin{Shaded}
\begin{Highlighting}[]
\NormalTok{urgency }\OtherTok{\textless{}{-}} \FunctionTok{get\_urgency}\NormalTok{(promises)}
\end{Highlighting}
\end{Shaded}

To be able to compare promises over time, across settings, or for
different politicians, they need to be organized into categories. That
is, what do these promises talk about. To do so, we extract objects from
promises by identifying the entities they refer to. We then select the
most common entities in texts, we call these subjects. Importantly, if
subjects are very similar, they are also merged. Once subjects are
gathered, we employ a keyword- assisted topic modelling to find related
terms to each of these subjects (Eshima, Imai, Sasaki 2023). This allows
that coherent topics are gathered that make sense considering the main
objects in promises. The number of subjects and related terms to these
subjects can be adjusted by users. The collection of subjects and
related terms is used to classify promises according to topic. However,
topics are assigned only when words related to subjects appear in
promise. This means if promises do not match any of the subjects and
related terms topics are not assigned. Promises can then be ranked
according to their urgency scores or aggregated into topics to get the
urgency of promises by topic. Users can rank the urgency of topics by
their sum, median, or mean.

\begin{Shaded}
\begin{Highlighting}[]
\NormalTok{subjects }\OtherTok{\textless{}{-}} \FunctionTok{extract\_subjects}\NormalTok{(promises)}
\NormalTok{related\_terms }\OtherTok{\textless{}{-}} \FunctionTok{extract\_related\_terms}\NormalTok{(promises, subjects)}
\NormalTok{urgency\_rank }\OtherTok{\textless{}{-}} \FunctionTok{get\_urgency\_rank}\NormalTok{(urgency)}
\end{Highlighting}
\end{Shaded}

\hypertarget{validation}{%
\subsection{Validation}\label{validation}}

\begin{itemize}
\tightlist
\item
  ROC plots (hand coding vs.~package)
\item
  Surveys for urgency rankings (Could we run an online survey with mturk
  maybe to check if people agree with poldis rankings?)
\end{itemize}

\hypertarget{limitations}{%
\subsection{Limitations}\label{limitations}}

Our approach to extracting promises and ranking preferences from
political discourses has two main limitations. First, we focus promises
in discourse but not on the build-up or context surrounding and
justifying the action. This makes it easier to rank preferences for
objects of policy and avoid deep interpretation while facilitating
reproducibility. Although we might miss some context, it makes sure we
are coding urgency related to a promise instead of a context that can be
used to justify many promises or no actions at all. Second, we do not
investigate present actions or past actions that will continue. While
present actions could arguably be considered urgent, they could also
reflect feasibility or easiness of political action (i.e.~low hanging
fruits). This choice of coding future promises only is both conceptual
and methodological. Conceptually, it is difficult to theoretically
distinct and compare the urgency of an action taken and an action that
has not yet been taken. Methodologically, it is easier to code future
actions proposed in discourse rather than present action since these can
come in the present and past tenses (e.g.~past continuous).

\end{document}
